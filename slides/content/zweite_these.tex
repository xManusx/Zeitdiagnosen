\section{Dezentralisierung als Antwort}
\quoteframe{The Automation of Society is Next, S.\ 103}{Dirk Helbing}{Rather than imposing a certain behavior in a top-down way, [digitally] assisted self-organization reaches efficient results by using the hidden forces, which determine the natural behavior of a complex dynamical system.}
\begin{frame}
	\frametitle{Selbstorganisation als Lösung}
	\note{Sorry für komische Übersetzungen}
	\begin{itemize}
		\item
			Selbsorganisation erzeugt resiliente Systeme
			\note{Wegen Diversity}
		\item
			Nicht gegen die \enquote{sozialen Kräfte}, sondern mit ihnen
		\item
			Ramenbedingungen müssen angepasst werden um stabile, vorteilhafte Systemzustände zu schaffen
		\item
			Schnellere, lokale Lösungen für Probleme, solange diese noch lokal sind
	\end{itemize}
\end{frame}

%\begin{frame}
	%\begin{itemize}
		%\item
			%Kooperatives Verhalten erfolgreicher als unkooperatives, wenn 
	%\end{itemize}

%\end{frame}


\begin{frame}
	\frametitle{Konkrete Vorschläge}
	\begin{itemize}[<+->]
		\item
			Externalitäten von Interaktionen (z.B. Reputation, Glück, Wohlstand, Emissionen, Abfall, Lärm...) können durch eine Einführung oder Modifikation von Feedback-Schleifen verändert werden
			\note{ökonomie: Geld, Wohlstand; soziale Systeme: Sanktionen, Anreize}

		\item[$\Rightarrow$] multi-dimensionale Feedback-Mechanismen nötig
		\item
			Beispielsweise durch Bewertungssysteme (\enquote{reputation systems})
			\note{Posten unter Klarnamen, Pseudonym oder anonym (aber vielleicht unterschiedlich gewertet?). Fakten, Meinungen und Werbung sollten unterscheidbar sein}
		\item
			Offene Informationssyteme (\enquote{Open Data Approach}) 
		\item
			Digitale Assistenten können, unterstützt mit Daten aus dem \textsc{IoT} und den offenen Informationssystemen, helfen, auf mögliche, positive Interaktionen hinzuweisen und vor negativen Interaktionen zu warnen
		\item
			Früher: Straßenbau für Industriegesellschaft, Schulen für Dienstleistungsgesellschaft --- jetzt: offene Informationssysteme für die Digitale Gesellschaft

	\end{itemize}
\end{frame}


%\begin{frame}
	%\frametitle{Stuff to pack into slides}
	%\begin{itemize}
		%\item
			%Austausch zwischen Firmen und Usern: win/win/win situation erzeugen
		%\item
			%Projektbasierte Ökonomie: Individuen finden sich zu Projekten zusammen, gründen aber keine Firma
	%\end{itemize}

%\end{frame}

%Humas are curious by nature, social, information-driven species. That's why explosion in data volume and processing power will transform society more fundamentally than other techs in the past
