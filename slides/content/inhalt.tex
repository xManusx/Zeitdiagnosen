\section{Zentrale Thesen}
\begin{frame}[label=complexity]
	\frametitle{Komplexität der Gesellschaft}
	Moderne Gesellschaften sind zu komplex um sie \textit{top-down} zu erfassen und zu organisieren.
\end{frame}

%\quoteframe{}{}{Die Gesellschaft ist zu komplex um sie top-down zu erfassen und zu organisieren}

%\quoteframe{}{}{Die Antwort auf eine immer komplexer werdende Welt ist Dezentralisierung und ein}
\begin{frame}[label=decentral]
	\frametitle{Dezentralisierung als Antwort}

	Die Antwort auf eine immer komplexer werdende Welt ist Dezentralisierung, unterstützt durch \hyperlink{collective_intel}{\textit{Kollektive Intelligenz}}.
\end{frame}

\begin{frame}[label=collective_intel]
	\frametitle{Kollektive Intelligenz}
	\begin{itemize}
		\item
			Verteilte, dezentrale Entscheidungsfindung
		\item
			Kein einfacher Mehrheitsentscheid
		\item
			Aber nicht notwendigerweise konsensbasiert
		\item
			Unter Einbeziehung einer Vielzahl von Lösungen oder Problemlösungsansätzen
	\end{itemize}
\end{frame}




%It is extremely important therefore to realize that the digital revolution is not  just about more powerful computers, better smartphones or fancier gadgets. The digital revolution will change all our personal lives, and it  will transform entire economies and societies. In fact, in the coming two  or  three  decades  we  will  see  some  dramatic  changes.  A  lot  of production and services will become automated, and this will fundamentally change the way we work in future.
%The explosion in data volumes, processing power, and Artifical Intelligence, known as the \enquote{digital revolution}
