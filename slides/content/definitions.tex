\section*{Definitionen und weiterführende Slides}
\begin{frame}[label=computational_soc]
	\begin{beamerboxesrounded}{Computational Social Science} 
		untersucht Sozial- und Verhaltensdynamiken mithilfe von sozialer Simulation, Netzwerkanalyse und Analyse sozialer Medien.\\
		{\large(Computational Social Science Society America, \url{https://computationalsocialscience.org/})}
	\end{beamerboxesrounded}
\end{frame}

\begin{frame}[label=complexity_science]
	%\frametitle{Complexity Science}
	\begin{beamerboxesrounded}{Complexity Science} 
		beschäftigt sich mit komplexen Systemen: Systemen aus vielen interagierenden Einzelteilen, deren Interaktionen ein globales Verhalten hervorbringen, dass nicht mehr trivial aus den Interaktionen  der Einzelteilen erklärt werden kann.\\
		{\large(Agents, Interaction and Complexity Group, University of Southampton, \url{http://www.complexity.ecs.soton.ac.uk/})}
	\end{beamerboxesrounded}
\end{frame}

\begin{frame}[label=game_theory]
	\begin{beamerboxesrounded}{Evolutionäre Spieltheorie} 
		Erforschung evolutionärer Prozesse, Ausbreitung und Verteilung von Verhaltensmustern in Tierpopulationen durch natürliche Selektion, Ausbreitung von Infektionen, mit Methoden und Modellen der Spieltheorie
	\end{beamerboxesrounded}

	\vspace{2\baselineskip}
	\begin{beamerboxesrounded}{Spieltheorie}
		Modellierung von Entscheidungssituationen in denen sich mehrere Beteiligte gegenseitig beeinflussen. Versucht dabei das rationale Entscheidungsverhalten in sozialen Konfliktsituationen davon abzuleiten.
	\end{beamerboxesrounded}
\end{frame}

\begin{frame}[label=agents]
	\begin{beamerboxesrounded}{Agentenbasierte Simulationen} 
		Simulation von vielen kleinen, autonomen Agenten, die meist nach simplen Regeln handeln um das daraus resultierende, komplexere System zu erforschen
	\end{beamerboxesrounded}
\end{frame}

\begin{frame}
	\frametitle{Wie müssen entsprechende Mechanismen für die Nutzung von \textit{Common Goods} aussehen?}
	\begin{enumerate}
		\item
			Klar definierte  In- und Outgroup
		\item
			Besitz und Nutzung von Allgemeingütern nach Regeln, die auf die lokalen Bedingungen zugeschnitten sind
		\item
			Kollektive Entscheidungsprozesse unter Einbeziehung der Agenten, die von der Nutzung jeweiligen Allgemeinguss betroffen sind 
		\item
			Versorgung und Nutzung kontrolliert von den Verwalter\gend{}innen, bzw. Personen die ihnen gegenüber Verantwortlich sind
		\item
			Sanktionen für die regelwidrige Nutzung
		\item
			Leicht zugängliche Konfliktlösung 
		\item
			Selbstverwaltung anerkannt von den höheren Hierarchiestufen
	\end{enumerate}
\end{frame}

\begin{frame}
	\frametitle{Das Data-Cord-Principle}
	\begin{itemize}
		\item
			Kombination aus (verschlüsseltem?) Personal Data Store
		\item
			und dem \enquote{Data-Cord-Principle}
		\item
			Inhalt der Datenspeicher wird mit jeweilige\gend{}r Besitzer\gend{}in oder Produzent\gend{}in verknüpft
		\item
			Besitzer kontrolliert Nutzungsrechte Dritter 
		\item
			Wird mittels Daten Profit erwirtschaftet, wird ein Teil davon an die Besitzer$^*$innen ausgezahlt
	\end{itemize}
\end{frame}

\begin{frame}
	\frametitle{Erfolgreiche Selbstorganisation des Verkehrs in Dresden}
	\begin{itemize}
		\item
			\enquote{Traditioneller} Ansatz: Daten sammeln, zentrale Stelle berechnet offline eine annähernd optimale Lösung
		\item
			Neuer Ansatz: Jede Ampel entscheidet lokal, wann sie schaltet
			\begin{itemize}
				\item
					Fahrzeiten der einzelnen Autos minimieren (\enquote{Egoistische} Entscheidung)
				\item
					Autoschlangen über einer bestimmten Länge dürfen sofort fahren (Entscheidung unter Einbeziehung von anderen)
			\end{itemize}
		\item
			Höherer Durchsatz/weniger Stau als top-down Ansatz und (rein) egoistischer Ansatz
		\item<2->
			Interessanter Gedanke: Könnte man Rezessionen als Staus in der Weltwirtschaft sehen? Wenn Kapital- oder Warenströme geblockt oder verzögert sind, könnte man ökonimische Verwerfungen mittels Echtzeitinformationen über Versorgungsnetze minimieren?
	\end{itemize}
\end{frame}
