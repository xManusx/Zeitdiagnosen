
\section{Komplexität der Gesellschaft}
\againframe{complexity}
\begin{frame}
	\frametitle{Steigende Komplexität}
	\begin{itemize}
		\item
			Welt ist ein stark gekoppeltes, dynamisches System
		\item
			Unscheinbare Handlungen, können fatale Folgen haben
			\note{Komponenten von komplexen, gekoppelten, dynamischen Systemen erzeugen selbstorganisiert neue Strukturen, Eigenschaften, Funktionen, z.B. stop-and-go Verkehr, riots}
			\note{Komplexe Systeme können außer Kontrolle geraten, auch wenn alle Beteiligten, gut informiert, gut ausgebildet ist, modernste Technologien verwendet und nur die besten Absichten hat}
		\item
			Demokratie ist zu langsam um mit Komplexität umgehen zu können
		\item
			Zu langsame Anpassung an neue Umstände führt zu Instabilität des Systems
			\note{Finanzkrise, immer erst reagieren, wenn es zu spät ist}
	\end{itemize}
\end{frame}
\begin{frame}
	\frametitle{Umgang mit steigender Komplexität}
	\begin{itemize}
		\item
			Zentralisierung: 
			Aggregation aller verfügbaren Daten, Erschaffung einer \enquote{magischen Kristallkugel}, die mithilfe von Big Data die Zukunft vorhersagt
		\item
			Was wenn Individuen anders entscheiden?

			$\Rightarrow$ Fehlende Flexibilität!
		%\item
			%Zwei Modelle damit umzugehen:
					%Deregulierung und Autoritarismus
					%\note{Deregulierung nicht stabil wegen Monopolen, Bankrott von Unternehmen}
					%\note{Autoritarismus nicht stabil, weil top-down Ansatz $\Rightarrow$ Komplexität steigt zu schnell}
		\item
			Bedarf nach sich flexibel anpassenden, belastbaren (\enquote{resilienten}) Systemen
		\item
			Antworten, wie solche Systeme aussehen und konstruiert werden könnten liefert \hyperlink{complexity_science}{\textit{Complexity Science}}
	\end{itemize}
\end{frame}

\begin{frame}
	\frametitle{Complexity Science in sozio-ökonomischen Systemen}
	\begin{itemize}
		\item
			\enquote{Verborgenen Kräfte}, die sozio-ökonomische Systeme bestimmen, können  mittels neuer Daten gemessen werden
		\item
			Wissen über \enquote{soziale Kräfte} erlaubt von \enquote{micro-level} Interaktionen auf \enquote{macro-level} Interaktionen zu schließen
		\item
			Beispiel: Fluiddynamische Theorie für Fußgänger
			\begin{itemize}
				\item
					Individuum agiert nach simplen Regeln
				\item
					Masse verhält sich scheinbar chaotisch
				\item
					Complexity Science kann Aussagen über Massen treffen
			\end{itemize}

	\end{itemize}
\end{frame}
